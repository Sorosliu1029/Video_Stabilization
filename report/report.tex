\documentclass[a4paper, twocolumn]{article}

\usepackage[sc]{mathpazo} % Use the Palatino font
\usepackage[T1]{fontenc} % Use 8-bit encoding that has 256 glyphs
\linespread{1.05} % Line spacing - Palatino needs more space between lines
\usepackage{microtype} % Slightly tweak font spacing for aesthetics

\usepackage[english]{babel} % Language hyphenation and typographical rules

\usepackage[hmarginratio=1:1,top=32mm,left=20mm,right=20mm,columnsep=20pt]{geometry} % Document margins
\usepackage[hang, small,labelfont=bf,up,textfont=it,up]{caption} % Custom captions under/above floats in tables or figures
\usepackage{booktabs} % Horizontal rules in tables

\usepackage{lettrine} % The lettrine is the first enlarged letter at the beginning of the text

\usepackage{enumitem} % Customized lists
\setlist[itemize]{noitemsep} % Make itemize lists more compact

\usepackage{abstract} % Allows abstract customization
\renewcommand{\abstractnamefont}{\normalfont\itshape\bfseries} % Set the "Abstract" text to bold
\renewcommand{\abstracttextfont}{\normalfont} % Set the abstract itself to small italic text

\usepackage{titlesec} % Allows customization of titles
\renewcommand\thesection{\Roman{section}} % Roman numerals for the sections
\renewcommand\thesubsection{\roman{subsection}} % roman numerals for subsections
\titleformat{\section}[block]{\large\scshape\centering}{\thesection.}{1em}{} % Change the look of the section titles
\titleformat{\subsection}[block]{\large}{\thesubsection.}{1em}{} % Change the look of the section titles

\usepackage{fancyhdr} % Headers and footers
\pagestyle{fancy} % All pages have headers and footers
\fancyhead{} % Blank out the default header
\fancyfoot{} % Blank out the default footer
\fancyhead[C]{\emph{Digital Image Processing 2016} Course Project Report}
\fancyfoot[RO,LE]{\thepage} % Custom footer text

\usepackage{titling} % Customizing the title section

\usepackage{hyperref} % For hyperlinks in the PDF
\hypersetup{hidelinks}

\usepackage{graphicx}

%----------------------------------------------------------------------------------------
%	TITLE SECTION
%----------------------------------------------------------------------------------------

\setlength{\droptitle}{-4\baselineskip} % Move the title up

\pretitle{\begin{center}\Huge\bfseries} % Article title formatting
	\posttitle{\end{center}} % Article title closing formatting
\title{Video Stabilization using Block Matching Algorithm} % Article title
\author{%
	\textsc{Liu Yang} \\[1ex] % Your name
	\normalsize Fudan University \\ % Your institution
	\normalsize \href{mailto:13307130167@fudan.edu.cn}{13307130167@fudan.edu.cn} % Your email address
}
\date{\today} % Leave empty to omit a date
\renewcommand{\maketitlehookd}{%
\begin{abstract}
%	\noindent
	\textbf{
		This report presents a video stabilization method based on block matching algorithm. It discusses a typical block matching algorithm -- EBMA, and compares EBMA with some faster block matching algorithm, along with algorithm implement results in MATLAB code.
	}
\end{abstract}
}
%----------------------------------------------------------------------------------------

\begin{document}
	
	% Print the title
	\maketitle
	
	%----------------------------------------------------------------------------------------
%		ARTICLE CONTENTS
	%----------------------------------------------------------------------------------------
	\section{Introduction}
	
	\lettrine[nindent=0em,lines=3]{V} ideo stabilization is a key problem in producing high quality video sequence, especially when we are in self-media age and much more videos are shot with smart phones, which means video stabilization is in great demand. \\
	Typically, there are three steps in video stabilization workflow: 
	
	\begin{itemize}
		\item motion estimation
		\item shaking recognition
		\item motion compensation
	\end{itemize}
	\noindent
	Among all three steps, the most computationally expensive and resource consumed one is motion estimation. Some mature models are discussed in \cite{vpc}: based on optical flow, based on pixel, based on block, based on mesh, etc. \\
	This report and video stabilization implement focus on block-based algorithm, start with naive and slow EBMA, to some other improved and faster algorithm. The algorithms that have been implemented are Exhaustive Block Matching Algorithm (EBMA), 2-D Log Search Method (a.k.a Diamond Search, DS), Three-Step Search Method (TSS). Multi-Resolution Motion Estimation with Hierarchical Block Matching Algorithm (HBMA) is also briefly introduced. Section II explains how these algorithms work. Section III makes a comparison between these algorithms and show video stabilization result. 
	
	%------------------------------------------------
	
	\section{Block Matching Algorithm}
	
	
	%------------------------------------------------
	
	\section{Comparison and Experiment}
	
	
	%------------------------------------------------
	
	\section{Conclusion}
	
	
	%----------------------------------------------------------------------------------------
	%	REFERENCE LIST
	%----------------------------------------------------------------------------------------
	
	\begin{thebibliography}{99} % Bibliography - this is intentionally simple in this template
		
		\bibitem[Video Processing and Communications, Chap. 6]{vpc}
		Wang Y, Ostermann J, Zhang Y Q. 
		\newblock {\em Video processing and communications[M]. Chap. 6: Two Dimensional Motion Estimation}  
		\newblock Upper Saddle River: Prentice Hall, 2002.
		
	\end{thebibliography}
	
	%----------------------------------------------------------------------------------------
	
\end{document}
