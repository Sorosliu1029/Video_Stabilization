\documentclass[a4paper, twocolumn]{article}

\usepackage[sc]{mathpazo} % Use the Palatino font
\usepackage[T1]{fontenc} % Use 8-bit encoding that has 256 glyphs
\linespread{1.05} % Line spacing - Palatino needs more space between lines
\usepackage{microtype} % Slightly tweak font spacing for aesthetics

\usepackage[english]{babel} % Language hyphenation and typographical rules

\usepackage[hmarginratio=1:1,top=32mm,left=20mm,right=20mm,columnsep=20pt]{geometry} % Document margins
\usepackage[hang, small,labelfont=bf,up,textfont=it,up]{caption} % Custom captions under/above floats in tables or figures
\usepackage{booktabs} % Horizontal rules in tables

\usepackage{lettrine} % The lettrine is the first enlarged letter at the beginning of the text

\usepackage{enumitem} % Customized lists
\setlist[itemize]{noitemsep} % Make itemize lists more compact

\usepackage{abstract} % Allows abstract customization
\renewcommand{\abstractnamefont}{\normalfont\itshape\bfseries} % Set the "Abstract" text to bold
\renewcommand{\abstracttextfont}{\normalfont} % Set the abstract itself to small italic text

\usepackage{titlesec} % Allows customization of titles
\renewcommand\thesection{\Roman{section}} % Roman numerals for the sections
\renewcommand\thesubsection{\roman{subsection}} % roman numerals for subsections
\titleformat{\section}[block]{\large\scshape\centering}{\thesection.}{1em}{} % Change the look of the section titles
\titleformat{\subsection}[block]{\large}{\thesubsection.}{1em}{} % Change the look of the section titles

\usepackage{fancyhdr} % Headers and footers
\pagestyle{fancy} % All pages have headers and footers
\fancyhead{} % Blank out the default header
\fancyfoot{} % Blank out the default footer
\fancyhead[C]{\emph{Digital Image Processing 2016} Course Project Report}
\fancyfoot[RO,LE]{\thepage} % Custom footer text

\usepackage{titling} % Customizing the title section

\usepackage{hyperref} % For hyperlinks in the PDF
\hypersetup{hidelinks}

\usepackage{graphicx}

%----------------------------------------------------------------------------------------
%	TITLE SECTION
%----------------------------------------------------------------------------------------

\setlength{\droptitle}{-4\baselineskip} % Move the title up

\pretitle{\begin{center}\Huge\bfseries} % Article title formatting
	\posttitle{\end{center}} % Article title closing formatting
\title{Video Stabilization using Block Matching Algorithm} % Article title
\author{%
	\textsc{Liu Yang} \\[1ex] % Your name
	\normalsize Fudan University \\ % Your institution
	\normalsize \href{mailto:13307130167@fudan.edu.cn}{13307130167@fudan.edu.cn} % Your email address
}
\date{\today} % Leave empty to omit a date
\renewcommand{\maketitlehookd}{%
\begin{abstract}
%	\noindent
	\textbf{
		This report presents a video stabilization method based on block matching algorithm. It discusses a typical block matching algorithm -- EBMA, and compares EBMA with some faster block matching algorithm, along with video stabilization implement result in MATLAB code.
	}
\end{abstract}
}
%----------------------------------------------------------------------------------------

\begin{document}
	
	% Print the title
	\maketitle
	
	%----------------------------------------------------------------------------------------
%		ARTICLE CONTENTS
	%----------------------------------------------------------------------------------------
	\section{Introduction}
	
	\lettrine[nindent=0em,lines=3]{V} ideo stabilization is a key problem in producing high quality video sequence, especially when we are in self-media age and much more videos are shot with smart phones, which means video stabilization is in great demand. \\
	Typically, there are three steps in video stabilization workflow: 
	
	\begin{itemize}
		\item motion estimation
		\item shaking recognition
		\item motion compensation
	\end{itemize}
	\noindent
	Among all three steps, the most computationally expensive and resource consumed one is motion estimation. Some mature models are discussed in \cite{vpc}: based on optical flow, based on pixel, based on block, based on mesh, etc. \\
	This report and video stabilization implement focus on block-based algorithm, start with naive and slow EBMA, to some other improved and faster algorithm. The algorithms that have been implemented are Exhaustive Block Matching Algorithm (EBMA), 2-D Log Search Method (a.k.a Diamond Search, DS), Three-Step Search Method (TSS). Multi-Resolution Motion Estimation with Hierarchical Block Matching Algorithm (HBMA) is also introduced. Section II explains how these algorithms work. Section III makes a comparison between the first three algorithms. Section IV presents a way to recognize shaking and shows the video stabilization result with motion compensation.
	
	%------------------------------------------------
	
	\section{Block Matching Algorithm}
	Considering motion estimation between two given frames, $ \psi(x,y, t_{1}) $ and $ \psi(x,y,t_{2}) $, we define motion vector (MV) at \textbf{x} between time $ t_{1} $ and $ t_{2} $ is the displacement of this point from $ t_{1} $ to $ t_{2} $. We will call the frame at time $ t_{1} $ the anchor frame, and the frame at time $ t_{2} $ the tracked frame. The anchor frame can be either before or after the tracked frame, as illustrated in Figure \ref{fig:backwards}. It is forward motion estimation when $ t_{1} < t_{2} $ while backward motion estimation when $ t_{1} > t_{2} $. We use $ \psi_{1}(\textbf{x}) $ to denote anchor frame and $ \psi_{2}(\textbf{x}) $ to denote tracked frame. \\
	In this report and the implement, we use backward motion estimation.
	\begin{figure}[h]
	\centering
	\includegraphics[width=1\linewidth]{backwards}
	\caption{Forward and backward motion estimation}
	\label{fig:backwards}
	\end{figure}

	\noindent	
	The idea behind motion estimation is to find the most similar part of two continuous frame, use the displacement of this two matched part as the motion vector.  Block Matching Algorithm (BMA) is a collection of block-based motion estimation algorithms.  BMA divide the whole video frame into non-overlapping small regions, usually called blocks, and assume that the motion within each block can be characterized by a simple parametric model, e.g., constant, affine or bilinear. If the block is sufficiently small, then we can get a quite accurate motion estimation. \\
	In the discussion below, $\Lambda$ will denote the whole video frame. And we divide a frame into $ M $ blocks, we use $ \beta_{ij} $ to represent the \textit{i}-th row, \textit{j}-th column block. So the block partition should satisfy: 
	\begin{displaymath}
		\cup  \beta_{ij} = \Lambda, \textrm{and} \quad \beta_{ij}  \cap \beta_{i'j'} = \emptyset, i \ne i', j \ne j' .
	\end{displaymath}
	In this report, the block is always square shape. And we assume the motion in each block is constant, i.e., the entire block is in same motion, which is called \textit{block-wise translational model}. \\
	The task of BMA is to find a single motion vector for each block. Given a block $ \beta $ in the anchor frame, motion estimation is to find a most matched block $ \beta' $ in the tracked frame so that the error between this two blocks is minimized. And the displacement vector \textbf{d} between this two blocks is the MV of the block in tracked frame. Since we simply adapt the block-wise translational model, estimated block can be represented as $ \beta_{es} = \beta_{an} + \textbf{d} $. So the error can be written as: 
	\begin{displaymath}
		\textit{E}(\textbf{d}_{m}, \forall 1 \leq m \leq M) = \sum_{m} \sum_{\beta \in \beta_{ij}} 
		| \psi_{2}(\beta + \textbf{d}_{m}) - \psi_{1}(\beta) | ^ {p}
	\end{displaymath}
	where $ \psi $ represents the gray value, $ \psi_{2} $ for tracked frame, $ \psi_{1} $ for anchor frame.\\
	Because all blocks in anchor frame is estimated independently, so the error minimizing problem is to minimize estimation error for each block, which is: 
	\begin{displaymath}
		\textit{E}_{m}(\textbf{d})_{M} = \sum_{\beta \in \beta_{ij}} | \psi_{2}(\beta + \textbf{d}_{m}) - \psi_{1}(\beta) | ^{p}
	\end{displaymath}
	To reduce computational load, the Mean Absolute Difference (MAD) error ( $ p = 1 $) is used in this report and the implement.
	\subsection{Exhaustive Block Matching Algorithm}
	Exhaustive Block Matching Algorithm (EBMA) is also called Full Search. It is the simplest but most computational expensive block matching algorithm of all.\\
	For a given block $ \beta $ in the anchor frame, EBMA searches surrounding area of $ \beta $, compares $ \beta $ with all candidate blocks $ \beta_{can} $ in the tracked frame and find best matched candidate block $ \beta' $ with minimum error. Then the displacement between anchor block $ \beta $ and best matched block $ \beta' $ is the estimated motion vector. \\
	In EBMA, search area can be parameterized by $ R_x $ pixels horizontally and $ R_y $ pixels vertically. Both $ R_x $ and $ R_y $ is symmetric with respect to the center of the anchor block $ \beta $. Usually the shaking is random either horizontally or vertically, so search area is a square area with $ R  = R_x = R_y $. \\
	EBMA can be illustrated in Figure \ref{fig:EBMA}. \\
	\begin{figure}[h]
	\centering
	\includegraphics[width=1\linewidth]{EBMA}
	\caption[EBMA figure]{The search procedure of EBMA for one anchor block}
	\label{fig:EBMA}
	\end{figure}
	
	\noindent
	For EBMA, the estimation accuracy can be specified by search step size, which is the distance between two nearby candidate blocks in the horizontal or vertical direction. 
	In this report and the implement, the same step size is used in both directions. In most case, step size is one pixel and is called \textit{integer-pel accuracy search}. On the other hand, for those low resolution frame, a more accurate motion estimation is needed. So in this case, we will use fractional-pel accuracy search. But there exists a problem that there may not be corresponding candidate points in the tracked frame for certain sample anchor points. To realize a step size of $ 1/K $ pixel search, the tracked frame firstly needs to be resized by factor $ K $. Typically the $ K = 2 $, which is know as \textit{half-pel accuracy search}. Half-pel accuracy search is shown in Figure \ref{fig:half-EBMA}.
	\begin{figure}[h]
	\centering
	\includegraphics[width=1\linewidth]{half-EBMA}
	\caption[half EBMA]{Half-pel accuracy block matching. Filled circles are samples existing in the original tracked frame, open circles are samples to be interpolated for calculating the matching error.}
	\label{fig:half-EBMA}
	\end{figure}
	
	\noindent
	Although EBMA can get a global optimal motion estimation and with half-pel accuracy search it can get more accurate, one obvious disadvantage of EBMA is that it requires intense computation when goes on with full search regardless of the possibility of candidate block to be the most matched block. \\
	Because of the computation problem, several faster algorithms have been developed to get a trade-off between the estimation accuracy and computation efficiency.
	
	\subsection{2-D Log Search Method}
	2-D Log Search Method is also known as Diamond Search. It is illustrated in Figure \ref{fig:ds}
	
	\begin{figure}[h]
	\centering
	\includegraphics[width=0.9\linewidth]{ds}
	\caption{2-D Log Search Method. Search points start with black circles, end with gray circles.}
	\label{fig:ds}
	\end{figure}

	\noindent
	It starts from the position corresponding to zero-displacement of the anchor block. In each step, nine nearby points within diamond shape will be tested to get a best matched point. Then the center of search area will move to the best matched point resulting from the previous step and repeat the step. In 2-D log search method, the search step size, also represents the radius of the search diamond shape, is reduced by one if the best matched point is the center point of the diamond, or on the border of the search area. Otherwise, the step size will remain same in next search step.
	The final step is reached when the step size reaches 1 pixel range. Usually the initial step size is set to be half of the max search range. \\
	As the fact that the number of steps and the total number of search points depends on the actual motion vectors, there is no limit to the number of 2-D log search steps, so it could find global minimum accurately while the computation expense goes down quickly.
	
	\subsection{Three-Step Search Method}
	As illustrated in Figure \ref{fig:tss}, the search starts with a step size equal to or slightly larger than half of the maximum search range. In each step, nine search points, which are in a square shape around the center of the previous search step, are compared. The step size is reduced by half after each step, and the search ends with step size reaches 1 pixel range. At each new step, the search center is moved to the best matched point resulting from the previous step.
	
	\begin{figure}[h]
	\centering
	\includegraphics[width=0.9\linewidth]{tss}
	\caption{Three-Step Search Method. Search points start with black circle, end with black square.}
	\label{fig:tss}
	\end{figure}
	
	\noindent
	Let $ R_{0} $ represent the initial search step size, there are at most $ L = \lfloor   \log_{2}{R_{0} + 1} \rfloor $ search step. At each search step, eight points are searched, except in the very first nine points searched. So the total number of search points is $ 8L + 1 $. For example, for a search range of $ R = 32 $, with EBMA, the total number of search points is $ (2*R+1)^2 = (2*32+1)^2 = 4225 $, whereas with three-step search method, the number is 41, ($ L = \lfloor   \log_{2}{R_{0} + 1} \rfloor = \lfloor   \log_{2}{\frac{32}{2} + 1} \rfloor = 5, 8*L+1 = 41$ ). Three-step search method is over 100 times efficient than EBMA.
	
	\subsection{Hierarchical Block Matching Algorithm}
	Among above motion estimation method, there two major difficulties when obtaining the correct solution:
	\begin{enumerate}
		\item the minimization function usually has many local minimum and it is not easy to reach the global minimum.
		\item the expense of computation is very high.
	\end{enumerate}
	To cope with above two problems, hierarchical block matching algorithm (HBMA) with multi-resolution approach is developed. It searches the optimal solution in successively finer resolutions, as illustrated in Figure \ref{fig:hbma}.
	
	\begin{figure}[h]
	\centering
	\includegraphics[width=1\linewidth]{hbma}
	\caption{3-level hierarchical block matching algorithm}
	\label{fig:hbma}
	\end{figure}

	\noindent
	First, we get pyramid-like representations of anchor frame and tracked frame. Each level is a resized frame as  a reduced resolution representation, where bottom level is the original one. The higher the level is, the lower its resolution is. Typical resolution reduction factor between two successive levels is 2, both horizontally and vertically.Then the motion vectors between corresponding levels of the two pyramids is estimated, starting from the top coarsest level and processing to the lower finer level repeatedly.\\
	Here we assume the same block size is used at different levels, so that the number of blocks is reduced by half in each dimension as well. Let the motion vector for block $ (i, j) $ as level $ l $ be denoted by $ \textbf{d}_{l, i, j} $. Starting from top level, we first find the motion vectors for all blocks in this level. At each lower level $ l > 1 $, for each block, its initial motion vector $ \widetilde{\textbf{d}}_{l, i, j} $ is interpolated from a corresponding block in level $ l - 1 $ by 
	\begin{displaymath}
		\widetilde{\textbf{d}}_{l, i, j} = 2*\textbf{d}_{l-1, \lfloor i/2 \rfloor, \lfloor j/2 \rfloor}
	\end{displaymath}
	Then within the search area, we search and get a correction vector $ \textbf{q}_{l, i, j} $, then make the correction to get the final estimated motion vector
	\begin{displaymath}
		\textbf{d}_{l, i, j} = \widetilde{\textbf{d}}_{l, i, j} + \textbf{q}_{l, i, j}
	\end{displaymath}
	We could notice that using a block with $ N $ size at level $ l $ corresponds to a block size of $ 2^{L-l} $ at the full resolution. Therefore, by using the same block size, serach range and step size at different levels, we actually use a larger block size, a larger search range and a larger step size in the beginning of the search, and then gradually half-reduce these three parameters in successive level steps.\\
	The benefit of HBMA is that by first searching the solution in a coarse resolution, one can usually obtain a solution that is close to the true motion vector. In addition, by limiting the search area to a small neighborhood of the solution obtained in the previous resolution, the total number of searches can be reduced, compared to those required by directly searching in the finest resolution over a large range.
	
	%------------------------------------------------
	
	\section{Comparison}
	
	
	%------------------------------------------------
	
	\section{Experiment}
		
		
	%------------------------------------------------
	
	\section{Conclusion}
	
	
	%----------------------------------------------------------------------------------------
	%	REFERENCE LIST
	%----------------------------------------------------------------------------------------
	
	\begin{thebibliography}{99} % Bibliography - this is intentionally simple in this template
		
		\bibitem{vpc}
		Wang Y, Ostermann J, Zhang Y Q. 
		\newblock {\em Video processing and communications[M]. Chap. 6: Two Dimensional Motion Estimation.}  
		\newblock Upper Saddle River: Prentice Hall, 2002.
		
		\bibitem{abmvf}
		Vella F, Castorina A, Mancuso M, et al. 
		\newblock {\em Digital image stabilization by adaptive block motion vectors filtering[J].}
		\newblock IEEE Transactions on Consumer Electronics, 2002, 48(3): 796-801.
		
		\bibitem{hbma}
		Bierling M. 
		\newblock {\em Displacement estimation by hierarchical blockmatching[C].}
		\newblock Visual Communications and Image Processing'88: Third in a Series. International Society for Optics and Photonics, 1988: 942-953.
		
	\end{thebibliography}
	
	%----------------------------------------------------------------------------------------
	
\end{document}
